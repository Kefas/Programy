\documentclass[a4paper, 12pt]{article}
\usepackage[utf8]{inputenc}
\usepackage[T1]{fontenc}
\usepackage[polish]{babel}
\usepackage{geometry}
\usepackage{amsmath}
\usepackage{amsfonts}


\begin{document}
Istnieje scisły zwiazek między rozkładem macierzy A na macierze L i U a metodą eliminacji Gaussa.Można wykazać, że elementy kolejnych kolumn macierzy L sa równe współczynnikom przez które mnożone są w kolejnych krokach wiersze układu równań celem dokonania eliminacji niewiadomych w odpowiednich kolumnach. Natomiast macierz U jest równa macierzy trójkątnej uzyskanej w eliminacji Gaussa.
\begin{equation*}
[A\vert b]=\left[ \begin{array}{llll}
2 & 2 & 4 & 4\\
1 & 2 & 2 & 4\\
1 & 4 & 1 & 1\\
\end{array}
\right] = \left[ \begin{array}{llrr}
2 & 2 & 4 & 4\\
0 & 1 & 0 & 2\\
0 & 4 & -1 & -1\\
\end{array}
\right]= \left[ \begin{array}{llrr}
2 & 2 & 4 & 4\\
0 & 1 & 0 & 2\\
0 & 0 & -1 & -7\\
\end{array}
\right]
\end{equation*}
\begin{equation*}
L=\left[ \begin{array}{lll}
1 & 0 & 0\\
\frac{1}{2} & 1 & 0\\
\frac{1}{2} & 3 & 1\\
\end{array} \right] \qquad
U=\left[ \begin{array}{llrr}
2 & 2 & 4 & 4\\
0 & 1 & 0 & 2\\
0 & 0 & -1 & -7\\
\end{array} \right] \\
\end{equation*}
\end{document}
